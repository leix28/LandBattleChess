
\documentclass[12pt]{article}

%设置页边距
\usepackage{geometry}
\geometry{left=2.5cm,right=2.5cm,top=2.5cm,bottom=2.5cm}

%地址链接支持
\usepackage{hyperref}

%支持分栏
\usepackage{multicol}

%页眉页脚
\usepackage{fancyhdr}
\pagestyle{fancy}
\lhead{Land Battle Chess Design Documentation}
\rhead{徐磊 2013011344}

%设置字体
\usepackage[no-math]{fontspec}
\usepackage{xunicode}
\usepackage{xltxtra}
\setmainfont[AutoFakeSlant]{Hiragino Sans GB W3}

%设置换行缩进
\usepackage[indentfirst=false,slantfont,boldfont]{xeCJK}
\usepackage{indentfirst}
\setlength{\parindent}{0.9cm}

%行距
\linespread{1.2}

\begin{document}

\title{Land Battle Chess 设计文档}

\author{徐磊\thanks{\href{mailto:leopard.lie@gmail.com}{leopard.lie@gmail.com}}\quad 2013011344}

\date{2014年9月}
\maketitle

\section{功能介绍}
Land Battle Chess是一个支持网络对战的军棋游戏。

\section{设计分析}
\subsection{网络通信}
网络通信部分使用socket API,在程序中使用EzServer与EzClient两个类进行了封装。这两个类都支持阻塞的发送消息与多线程的接受消息,利用QT的信号槽机制来实现线程之间的通信。EzServer除了支持收发消息外,还提供了创建监听的函数。EzClient则包括了连接服务器的函数。
\subsection{棋盘模型}
棋盘模型是单独的ChessModel类,这个类包括了双方玩家共50个棋子的坐标,以及双方玩家对对方棋子的猜测信息。通过在网络上直接传递ChessModel来实现信息的交互。ChessModel可以根据玩家的信息生成用于绘制的棋盘状态(包括自动旋转棋盘,使得用户位于屏幕的下方)。支持判定胜负,设置初始局面,判断是否可以移动,处理旗子移动等。
\subsection{棋盘绘制}
棋盘绘制类ChessBoard继承自QLabel,可以根据棋盘模型生成的绘制数据进行绘制。另外,ChessBoard还监控鼠标点击事件,将对应的旗子坐标以消息的形式发送给控制器。
\subsection{控制器}
MainWindow除了包含一个界面用来放置棋盘和其它必要的按钮外,主要的功能是控制器。在Mainwindow中,需要将创建服务器、连接服务器的消息关联到EzServer、EzClient恰当的函数上,根据EzServer、EzClient接收的数据来更新本地的ChessModel,并将ChessModel生成的绘制信息发送给ChessBoard。MainWindow需要处理ChessBoard发送的按键信号,以及自己保存的历史按键信息,来做出正确的操作。另外,MainWindow中还包括一套复杂的计时机制。
\subsection{其它}
根据要求,需要实现一个奇怪的输入IP的Dialog,写作MyInputDialog。需要实现用户信息的传递,包装了User类包括ID和图片,UserWidget用于显示User信息。
\section{通信协议}
系统自动将服务器端设为玩家A,客户端设为玩家B。一条有效地通信长度永远是sizeof(ChessModel)+1,第一个字符若为A或B,表面A玩家或B玩家走了一步,之后的sizeof(ChessModel)的长度为ChessModel。若第一个字符为a或b,表示A玩家或B玩家完成了棋子的布局,等待开始游戏,之后有一个ChessModel来表明棋盘的状态。若第一个字符为T,表示求和,而第一个字符为t表示同意求和。第一个字符为-1表示认输。

发送用户名和头像的协议,首字母为U表示发送用户名,之后sizeof(ChessModel)的长度为用户名(默认用户名少于该大小,用0补足长度)。首字母为I表示与头像相关的信息,首先发送只包含5个字节有效信息的数据,用0补足长度,该信息的后四位是一个整数表示图片的大小。随后以I开头的若干条数据来发送该图片。
\section{总结}
Land Battle Chess基本将数据、图形和控制分开,是程序的耦合性较小。
\end{document}